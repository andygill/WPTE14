\documentclass[a4paper,UKenglish]{oasics}
%This is a template for producing OASIcs articles.
%See oasics-manual.pdf for further information.
%for A4 paper format use option "a4paper", for US-letter use option "letterpaper"
%for british hyphenation rules use option "UKenglish", for american hyphenation rules use option "USenglish"

\usepackage{microtype}%if unwanted, comment out or use option "draft"

%\graphicspath{{./graphics/}}%helpful if your graphic files are in another directory

\bibliographystyle{plain}% the recommended bibstyle

% Author macros::begin %%%%%%%%%%%%%%%%%%%%%%%%%%%%%%%%%%%%%%%%%%%%%%%%
\title{HERMIT: An Equational Reasoning Model to Implementation Rewrite System for Haskell\footnote{This material is based upon work supported by the National Science Foundation under Grant No. 1117569.}}
\titlerunning{HERMIT: An Equational Reasoning Rewrite System} %optional, in case that the title is too long; the running title should fit into the top page column

\author[1]{Andrew Gill}
\affil[1]{Information Technology and Telecommunication Center,\\
Department of Electrical Engineering and Computer Science,\\
The University of Kansas, USA\\
\texttt{andygill@ku.edu}}
\authorrunning{A.\,Gill}%mandatory. First: Use abbreviated first/middle names. Second (only in severe cases): Use first author plus 'et. al.'

\Copyright{Andrew Gill}%mandatory. OASIcs license is "CC-BY";  http://creativecommons.org/licenses/by/3.0/

\subjclass{D.3.4 Translator writing systems and compiler generators}% mandatory: Please choose ACM 1998 classifications from http://www.acm.org/about/class/ccs98-html . E.g., cite as "F.1.1 Models of Computation". 
\keywords{Program Transformation, Equational Reasoning}% mandatory: Please provide 1-5 keywords
% Author macros::end %%%%%%%%%%%%%%%%%%%%%%%%%%%%%%%%%%%%%%%%%%%%%%%%%

%Editor-only macros:: begin (do not touch as author)%%%%%%%%%%%%%%%%%%%%%%%%%%%%%%%%%%
\serieslogo{}%please provide filename (without suffix)
\volumeinfo%(easychair interface)
  {Billy Editor and Bill Editors}% editors
  {2}% number of editors: 1, 2, ....
  {Conference/workshop/symposium title on which this volume is based on}% event
  {1}% volume
  {1}% issue
  {1}% starting page number
\EventShortName{}
\DOI{10.4230/OASIcs.xxx.yyy.p}% to be completed by the volume editor
% Editor-only macros::end %%%%%%%%%%%%%%%%%%%%%%%%%%%%%%%%%%%%%%%%%%%%%%%

\begin{document}

\maketitle

\begin{abstract}
HERMIT is a rewrite system for Haskell.
%
Haskell, a pure functional programming language,
is an ideal candidate for performing equational reasoning.
%
Equational reasoning, replacing equals with equals,
is a tunneling mechanism between different, but equivalent, programs.
%
The ability to be agile in representation and implementation,
but retain equivalence, brings many benefits. Post-hoc 
optimizations are one obvious application of representation
agility, and one we would like to explore.

%\end{abstract}

What we want to explore is the mechanization of rewriting, 
inside real Haskell programs, enabling the prototyping
of new optimizations, the explicit use of types to direct
transformations, and perform larger data refinement tasks
than are currently undertaken.
%
Paper and pencil program transformations have been written
that improve performance in a principled way; indeed some
have turned the act of program transformation into an art form.
But there is only so far a paper and pencil can take you.
%
There are also source code development environments
that provide support for refactoring, such as HaRe.
These work at the syntactical
level, and Haskell is a large and complex language.

In this talk, we overview HERMIT, the Haskell
equational reasoning model to implementation tunnel.
%
HERMIT operates at the Glasgow Haskell's Core level,
deep inside GHC, where type information is easy to
obtain, and the language being rewritten is smaller.
%
HERMIT provides three levels of support for
transformation and prototyping: a strategic programming
base with many typed rewrite primitives, a simple
shell that can used to interactively request rewrites
and explore transformation possibilities, and a batch
language that can mechanize focused, and optionally
program specific, optimizations.
%
We will demonstrate all three of these levels, and show how they cooperate.

The explicit aim of the HERMIT project is to explore data refinement,
and the worker/wrapper transformation, a specific way of mechanizing
data refinement.
%
HERMIT has been successfully used on toy examples, efficient \verb|reverse|,
tupling-\verb|fib|, and many other examples from the literature.
We will show two larger and more interesting examples of program transformation using HERMIT.
Specifically, we will show the mechanization of 
the making a century program refinement pearl, originally by Richard Bird,
and the exploration of datatype alternatives in Graham Hutton's 
implementation of John Conway's Game of Life.
\end{abstract}

\end{document}
